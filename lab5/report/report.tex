\documentclass[12pt, a4paper, oneside]{ctexart}
\usepackage{amsmath, amsthm, amssymb, graphicx}
\usepackage{hyperref}
\usepackage{listings}
\usepackage{xcolor}
\usepackage{color}
\usepackage{enumerate}
\usepackage{epstopdf}
\usepackage{float}
\usepackage{framed}
\usepackage{titlesec}
\usepackage{cite}
\usepackage{longtable}
\usepackage{booktabs}
\usepackage{geometry}
\usepackage{caption}
\usepackage{subfigure}
\usepackage[section]{placeins}
\usepackage{fancyhdr}%导入fancyhdr包
\usepackage{ctex}%导入ctex包
\usepackage[ruled,vlined]{algorithm2e}
\hypersetup{
    colorlinks=true,
    linkcolor=blue,
    filecolor=blue,      
    urlcolor=blue,
    citecolor=cyan,
}
\definecolor{dkgreen}{rgb}{0,0.6,0}
\definecolor{gray}{rgb}{0.5,0.5,0.5}
\definecolor{mauve}{rgb}{0.58,0,0.82}
\definecolor{shadecolor}{rgb}{0.5,0.5,0.5}
\lstset{ %
    language=Python,                % the language of the code
    basicstyle=\footnotesize,           % the size of the fonts that are used for the code
    numbers=left,                   % where to put the line-numbers
    %numberstyle=\tiny\color{gray},  % the style that is used for the line-numbers
    %stepnumber=2,                   % the step between two line-numbers. If it's 1, each line 
                            % will be numbered
    %numbersep=5pt,                  % how far the line-numbers are from the code
    %backgroundcolor=\color{blue},      % choose the background color. You must add \usepackage{color}
    showspaces=false,               % show spaces adding particular underscores
    %showstringspaces=false,         % underline spaces within strings
    showtabs=false,                 % show tabs within strings adding particular underscores
    frame=single,                   % adds a frame around the code
    rulecolor=\color{black},        % if not set, the frame-color may be changed on line-breaks within not-black text (e.g. commens (green here))
    tabsize=2,                      % sets default tabsize to 2 spaces
    captionpos=b,                   % sets the caption-position to bottom
    breaklines=true,                % sets automatic line breaking
    breakatwhitespace=false,        % sets if automatic breaks should only happen at whitespace
    % title=\lstname,                   % show the filename of files included with \lstinputlisting;
                            % also try caption instead of title
    keywordstyle=\color{blue},          % keyword style
    commentstyle=\color{dkgreen},       % comment style
    stringstyle=\color{mauve},         % string literal style
    escapeinside={\%*}{*)},            % if you want to add LaTeX within your code
    morekeywords={*,...}               % if you want to add more keywords to the set
}
\title{数据分析及实践\_Assignment5}
\author{Xiaoma}
\date{\today}
\pagestyle{fancy}

\lhead{第 5 次实验\\\today}
\chead{中国科学技术大学\\数据分析及实践}

\rhead{Assignment 5\\ {\CTEXoptions[today=old]\today}}
\newcommand{\upcite}[1]{\textsuperscript{\cite{#1}}}
\begin{document}
\maketitle
\section{实验要求}
在实验三实现的数据分析的基础上使用\textbf{PISA2018}数据集
,对\textbf{REPEAT}列进行分类。
\begin{itemize}
    \item 实现至少一种分类算法(例如:决策树、KNN、朴素贝叶斯、感知机和集成算法等)
    \item 参考实验三中的特征工程,测试算法在\textbf{PISA2018}数据集上的预测性能,并撰写实验报告
    \item 实验报告需记录最终的方案
\end{itemize}

具体要求:
\begin{itemize}
    \item 代码实现可以使用现有的机器学习库,也可以自行编写实现算法
    \item 预测任务与实验三一致,以\textbf{ACC}为评价指标
    \item 使用5折交叉验证的方法测试模型性能
\end{itemize}
\newpage

\section{实验环境}
\textbf{VSCode + Python3.9.13}
\section{实验步骤}
\subsection{数据预处理}
首先根据实验要求,去掉相关性最高的5个特征,已知实验三的实验结果,抽取剩余与$REPEAT$最相关的4个特征以及\textbf{REPEAT}列,
去除缺失\textbf{REPEAT}的样本,然后观察其余缺失值,通过观察可以发现
有缺失值的特征\textbf{HOMEPOS}
的缺失数量相对于数据总体很小,故为了保证数据准确性,将缺失值直接删去。

\subsubsection{数据去噪}
分别使用了$3\sigma$准则与\textbf{EllipticEnvelope}方法进行数据去噪,
但通过后面对实验结果的观察可以发现这两种去噪方法对模型性能
的影响几乎没有差别。

\subsection{模型训练}
使用\textbf{sklearn.model\_selection}包中的\textbf{cross\_val\_score}方法直接进行$n$
折交叉验证。

划分特征数据与\textbf{REPEAT}标签,分别使用\textbf{XGBClassifier,
MLPClassifier,DecisionTreeClassifier,KNeighborsClassifier}模型来进行训练,
经过反复调参,最终得到的最佳性能分别为

\begin{table*}[h]
    \centering%把表居中
    \begin{tabular}{cc}%内容全部居中
    \toprule%第一道横线
    模型&5折交叉验证性能 \\
    \midrule%第二道横线 
    \textbf{XGBClassifier} &0.8203\\
    \textbf{DecisionTreeClassifier} &0.8240\\
    \textbf{MLPClassifier} &0.7651\\
    \textbf{KNeighborsClassifier} & 0.7481\\

    \bottomrule%第三道横线
    \end{tabular}
\end{table*}

根据实验结果可以推测数据引入的特征量过少,根据实验三得到的相关度系数,我们选择对不同的模型使用
不同的特征组合:
\begin{itemize}
    \item \textbf{XGBClassifier} \\ 
    \textbf{ISCEDL,AGE,HOMEPOS,COBN\_F,ENTUSE,PROGN,\\
    COBN\_M,IC001Q09TA}
    \item \textbf{DecisionTreeClassifier}\\
    \textbf{COBN\_F,ST011D17TA,ENTUSE,HOMEPOS\\
    IC008Q02TA,IC001Q11TA,MISCED\_D,SCREADDIFF}
    \item \textbf{MLPClassifier}\\
    \textbf{ISCEDL,AGE,HOMEPOS,COBN\_F,ENTUSE,PROGN,\\
    COBN\_M,IC001Q09TA}
    \item \textbf{KNeighborsClassifier}\\
    \textbf{ISCEDL,AGE,HOMEPOS,COBN\_F,ENTUSE,PROGN,\\
    COBN\_M,IC001Q09TA} 
\end{itemize}


通过反复调参,
最终得到的最佳性能分别为

\begin{table*}[h]
    \centering%把表居中
    \begin{tabular}{cc}%内容全部居中
    \toprule%第一道横线
    模型&5折交叉验证性能 \\
    \midrule%第二道横线 
    \textbf{XGBClassifier} &0.8391\\
    \textbf{DecisionTreeClassifier} &0.8302\\
    \textbf{MLPClassifier} &0.8301\\
    \textbf{KNeighborsClassifier} & 0.7881\\

    \bottomrule%第三道横线
    \end{tabular}
\end{table*}
\clearpage
取四个模型所有特征的并集,通过反复调参,
最终得到的最佳性能分别为

\begin{table*}[h]
    \centering%把表居中
    \begin{tabular}{cc}%内容全部居中
    \toprule%第一道横线
    模型&5折交叉验证性能 \\
    \midrule%第二道横线 
    \textbf{XGBClassifier} &0.8375\\
    \textbf{DecisionTreeClassifier} &0.8316\\
    \textbf{MLPClassifier} &0.8301\\
    \textbf{KNeighborsClassifier} & 0.7999\\

    \bottomrule%第三道横线
    \end{tabular}
\end{table*}

由结果可知,模型性能已几乎不变。

考虑将所有特征全部使用,最终得到的最佳性能分别为

\begin{table*}[h]
    \centering%把表居中
    \begin{tabular}{cc}%内容全部居中
    \toprule%第一道横线
    模型&5折交叉验证性能 \\
    \midrule%第二道横线 
    \textbf{XGBClassifier} &0.6869\\
    \textbf{DecisionTreeClassifier} &0.8109\\
    \textbf{MLPClassifier} &0.8013\\
    \textbf{KNeighborsClassifier} & 0.3872\\

    \bottomrule%第三道横线
    \end{tabular}
\end{table*}

通过观察可知,使用除了最初去除的5个特征以外的全部特征进行训练,得到的模型性能要比使用子集差,则
我们选择\textbf{COBN\_F,OCOD2, ISCEDL,AGE,\\HOMEPOS,ENTUSE,PROGN,COBN\_M,IC001Q09TA,
ST011D17TA,\\IC008Q02TA,IC001Q11TA,MISCED\_D,SCREADDIFF}作为最终选择的训练参数。

\section{总结}

通过观察实验结果可以发现,在训练过程中,加入与预测标签不相关的特征对预测任务无积极效果。

观察使用的特征含义:
\begin{table*}[h]
    \centering%把表居中
    \begin{tabular}{cc}%内容全部居中
    \toprule%第一道横线
    特征&含义 \\
    \midrule%第二道横线 
    \textbf{COBN\_F} &Country of Birth National Categories- Mother\\
    \textbf{COBN\_M} &Country of Birth National Categories- Father\\
    \textbf{OCOD\_2} &ISCO-08 Occupation code - Father\\
    \textbf{AGE} &AGE\\
    \textbf{HOMEPOS} &Home possessions (WLE)\\
    \textbf{ENTUSE} &ICT use outside of school (leisure) (WLE)\\
    \textbf{PROGN} &Unique national study programme code\\
    \textbf{IC001Q09TA} &Available for you to use at home: Printer\\
    \textbf{ST011D17TA} &In your home: <Country-specific wealth item 1>\\
    \textbf{IC008Q02TA} &Use digital devices outside of school: Playing collaborative online games.\\
    \textbf{IC001Q11TA} &Available for you to use at home: <ebook reader>, e.g. <Amazon Kindle>\\
    \textbf{MISCED\_D} &Mothers Education - alternate definition (ISCED)\\
    \textbf{SCREADDIFF} &Self-concept of reading: Perception of difficulty (WLE)\\
    
    \bottomrule%第三道横线
    \end{tabular}
\end{table*}
可以发现复读还与家庭的经济状况,父母的职业与教育程度,个人是否沉迷与网络有关。

\end{document}